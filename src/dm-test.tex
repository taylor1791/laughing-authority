\documentclass{dm}

\newcommand{\theversion}{v0.0.0}
\title{DM Class Guide}

\begin{document}
  \begin{cols}

    \killsectionspace{}
    \section{Kobols}
    Kobolds are often dismissed as cowardly, foolish, and weak, but these little reptilian creatures actually have a strong social structure that stresses devotion to the tribe, are clever with their hands, and viciously work together in order to overcome their physical limitations.

    \begin{content}[Urds]
      Winged kobolds, known as urds, hatch seemingly at random from kobold eggs, even in a tribe that as no adult urds.
      Although being able to fly is an incredible gift, and it would be expected for kobolds to interpret the wings as a blessing from Tiamat, ordinary kobolds resent urds and don't get along with them.
    \end{content}

    \subsection{Tactics}
    Because they are physically weak individuals, kobolds know they have to use superior numbers and cunning to take down powerful foes.
    In addition to their Pack Tactics trait described in the \textit{Monster Manual}, they use traps, ambushes, terrain, allied monsters, and any other advantage they can squeeze out of their environment.
    Essentially, the only way kobolds can win is not to play fair.

    Standard kobold tactics include the following:
    \begin{itemize}
      \item Attacking light sources to extinguish them, so kobolds can use their darkvision to best advantage.
      \item Using hit-and-run maneuvers, fleeing between attacks to better or more secure vantage points.
            Often their goal is to attract enemies and draw the foes into greater danger, which can especially effective if the invaders have made camp, are injured, or are otherwise compromised (such as having to move by climbing or swimming).
    \end{itemize}

    \subsection{Kobold Lairs}
    The lair of a kobold tribe is usually a maze of twisty little passages, sometimes stretching for hundreds of yards, and frequently guard by traps.
    The area has a host of intersections, abrupt dead-ends, tunnels that cross over or under one another, concealed passages, and other features that make the lair difficult for outsiders to navigate.
    Creatures larger than a kobold have to squat or crawl single file in order to fit through the tunnels of a kobold warren, 

    Kobold warrens are usually composed of the following:
    \begin{enumerate}[twocol]
      \item escape tunnels
      \item mushroom farms
      \item throne room
      \item traps
    \end{enumerate}

    \subsection{Roleplaying a Kobold}
    A kobold acknowledges its weakness in the face of a hostile world.
    It nows it is puny, bigger creatures will exploit it, it will probably die at a young age, and its life will be full of toil.
    Although this outlook seems bleak, a kobold finds satisfaction in its work, the survival of its tribe, and the knowledge that it shares a heritage with the mightiest of dragons.

    \subsubsection{Kobold Traits}
    Your Kobold character has the following racial traits.

    \paragraph{Ability Score Increase}
    Your Dexterity score increases by 2, and your Strength score is reduced by 2.

    \paragraph{Size}
    Kobolds are between 2 and 3 feet tall and weigh between 25 and 35 pounds.
    Your size is small.
    \calc{Size}{2d4}
    \calc{Height}{2'1'' + Size}
    \calc{Weight}{25 lb. + Size}

    \paragraph{Grovel, Cower, and Beg}
    As an action on your turn, you can cower pathetically to distract nearby foes.
    Until the end of you next turn, your allies again advantage on attack rolls against enemies within 10 feet of you that you can see you.
    Once you use this trait, you can't use it again until you finish a short or long rest.

    \paragraph{Pack Tactics}
    You have advantage on an attack roll against a creature if at least one of your allies is within 5 feet of the creature and the ally isn't incapacitated.

    \paragraph{Sunlight Sensitivity}
    You have disadvantage on attack rolls and on Wisdom (Perception) checks that rely on sight when you, the target of your attack, or whatever you are trying to perceive is in direct sunlight.

    \subsection{Physical Variations}
    Kobolds vary widely in how their scales are colored and patterend.
    Most kobolds of the same tribe tend to have similar coloration.

    \subparagraph{Color and Pattern}
    Randomly select from these colors: black, blue, brown, gray, green, orange, orange-brown red, red-brown, tan, white, or patterned.
    If patterend, roll on the pattern table and twice from the colors ignoring duplicates and patterns.

    \begin{tabularz}[Scale Pattern]
      \begin{tabularx}{\linewidth}{cX}
        \textbf{d20}  & \textbf{Scale Pattern} \\
        01--04 & Mottled\\
        05--08 & Reticulated \\
        09--12 & Shaded \\
        13--16 & Spotted \\
        17--20 & Striped \\
      \end{tabularx}
    \end{tabularz}
  \end{cols}
\end{document}
